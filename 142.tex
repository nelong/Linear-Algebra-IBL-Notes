\documentclass{exam}
\usepackage[all]{xy}
\usepackage{color,graphicx}
\usepackage{esvect}
\usepackage{amsfonts, amssymb,amsmath}
\usepackage{enumitem}

\makeatletter
\renewcommand*\env@matrix[1][*\c@MaxMatrixCols c]{%
  \hskip -\arraycolsep
  \let\@ifnextchar\new@ifnextchar
  \array{#1}}
\makeatother

\newcommand\Rn{$\mathbb{R}^n$}
\newcommand\Rm{$\mathbb{R}^m$}
\newcommand\R{$\mathbb{R}$}
\newcommand\bq{\begin{question}}
\newcommand\eq{\end{question}}
\newcommand\be{\begin{enumerate}}
\newcommand\ee{\end{enumerate}}
\renewcommand{\labelenumi}{\alph{enumi})}
\newcommand*\colvec[1]{
        \global\colveccount#1
        \begin{bmatrix}
        \colvecnext
}
\def\colvecnext#1{
        #1
        \global\advance\colveccount-1
        \ifnum\colveccount>0
                \\
                \expandafter\colvecnext
        \else
                \end{bmatrix}
        \fi
}

\begin{document}


\textbf{Question 142}: Show that the set $\{1+t,t+t^2,1+t^3,t+t^2+t^3\}$ spans all of $\mathbb{P}_3$.\newline
\vspace{0.2in}
\newline

\textit{Proof}: Let $S=\{1+t,t+t^2,1+t^3,t+t^2+t^3\}$. In order to show that $S$ spans $\mathbb{P}_3$, we must show that $span(S)=\mathbb{P}_3$. We will show these sets are equal by showing that $span(S)\subseteq\mathbb{P}_3$ and $\mathbb{P}_3\subseteq span(S)$. 

Now, $span(S)\subseteq\mathbb{P}_3$ since an element of $span(S)$ has the form $$c_1(1+t)+c_2(t+t^2)+c_3(1+t^3)+c_4(t+t^2+t^3)=(c_1+c_3)+(c_1+c_2+c_4)t+(c_2+c_4)t^2+(c_3+c_4)t^3$$ which is in $\mathbb{P}_3$ by definition of the set. So it remains to show that $\mathbb{P}_3\subseteq span(S).$ Consider an arbitrary element of $\mathbb{P}_3$, $f(t)=a_0+a_1 t+a_2 t^2+a_3 t^3.$ We will show that $f(t)\in span(S).$ We need to know whether the equation $$c_1(1+t)+c_2(t+t^2)+c_3(1+t^3)+c_4(t+t^2+t^3)=a_0+a_1 t+a_2 t^2+a_3 t^3$$ has a solution. Again, we can rewrite this equation in the form $$(c_1+c_3)+(c_1+c_2+c_4)t+(c_2+c_4)t^2+(c_3+c_4)t^3=a_0+a_1 t+a_2 t^2+a_3 t^3.$$ These polynomials are equal exactly when coefficients of corresponding terms are equal, so we have the system of linear equations 
$$c_1+c_3=a_0$$
$$c_1+c_2+c_4=a_1$$
$$c_2+c_4=a_2$$
$$c_3+c_4=a_3$$
with corresponding augmented matrix 
$$\begin{bmatrix}[cccc|c]
  1 & 0 & 1 & 0 & a_0\\
  1 & 1 & 0 & 1 & a_1\\
  0 & 1 & 0 & 1 & a_2\\  
  0 & 0 & 1 & 1 & a_3\\
\end{bmatrix}$$
Since the reduced row-echelon form of the \textit{coefficient} matrix is 
$$\begin{bmatrix}[cccc]
  1 & 0 & 0 & 0 \\
  0 & 1 & 0 & 0 \\
  0 & 0 & 1 & 0 \\  
  0 & 0 & 0 & 1 \\
\end{bmatrix}$$
there is no pivot in the augmentation column. By the consistency theorem, the corresponding system of linear equations is consistent, and $f(t)$ can always be written as a linear combination of the elements in $S$. By definition of \textit{span}, $f(t)\in span(S)$ and therefore $\mathbb{P}_3\subseteq span(S)$. Hence, $\mathbb{P}_3= span(S)$ and $S$ spans $\mathbb{P}_3$.
\end{document}