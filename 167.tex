\documentclass{exam}
\usepackage[all]{xy}
\usepackage{color,graphicx}
\usepackage{esvect}
\usepackage{amsfonts, amssymb,amsmath}
\usepackage{enumitem}

\makeatletter
\renewcommand*\env@matrix[1][*\c@MaxMatrixCols c]{%
  \hskip -\arraycolsep
  \let\@ifnextchar\new@ifnextchar
  \array{#1}}
\makeatother

\newcommand\Rn{$\mathbb{R}^n$}
\newcommand\Rm{$\mathbb{R}^m$}
\newcommand\R{$\mathbb{R}$}
\newcommand\bq{\begin{question}}
\newcommand\eq{\end{question}}
\newcommand\be{\begin{enumerate}}
\newcommand\ee{\end{enumerate}}
\renewcommand{\labelenumi}{\alph{enumi})}
\newcommand*\colvec[1]{
        \global\colveccount#1
        \begin{bmatrix}
        \colvecnext
}
\def\colvecnext#1{
        #1
        \global\advance\colveccount-1
        \ifnum\colveccount>0
                \\
                \expandafter\colvecnext
        \else
                \end{bmatrix}
        \fi
}

\begin{document}


\textbf{Question 167}: Prove that if $T$ is a linear transformation from $V$ to $W$, then $T(\Vec{0_V})=\Vec{0_W}$.\newline
\vspace{0.2in}
\newline

\textit{Proof}: Let $V$ and $W$ be vector spaces and $T$ be a linear transformation from $V$ to $W$. We will show that $T(\Vec{0}_V)=\Vec{0}_W$. 

Let $\Vec{v}\in V$. We have previously shown that $0\Vec{v}=\Vec{0}_V$ and the analogous fact is true in the vector space $W$.\footnote{see Dr. Long's proof of 115(e)} For convenience, we reproduce this proof here. \newline
\vspace{0.1in}
\newline

\textbf{Claim}: For any vector $\Vec{u}$ in a vector space $U$, $0\Vec{u}=\Vec{0}_U$.\newline
\vspace{0.1in}
\newline

\textit{Proof of claim}: By vector space properties of $U$, we have $$\Vec{u}=1\Vec{u}=(1+0)\Vec{u}=1\Vec{u}+0\Vec{u}=\Vec{u}+0\Vec{u}.$$     Since $U$ is a vector space and $\Vec{u}\in U,$ we have the additive inverse vector $-\Vec{u}\in U$. Again by vector space properties, $$\Vec{0}_U=(-\Vec{u})+\Vec{u}=(-\Vec{u})+[\Vec{u}+0\Vec{u}]=[(-\Vec{u})+\Vec{u}]+0\Vec{u}=\Vec{0}_U+0\Vec{u}=0\Vec{u}.$$
\vspace{0.1in}
\newline

We now return to the proof of 167. Since $T$ is linear, it follows that $$T(\Vec{0}_V)=T(0\Vec{v})=0\cdot T(\Vec{v})=\Vec{0}_W,$$ as desired.

\end{document}

The following approaches are problematic because it is difficult to show that $T(\Vec{0_V})$ in all of $W$; it is straightforward to show that $T(\Vec{0_V})$ acts as a zero vector in $range(T)$ but it does not immediately follow that it acts as the zero vector in the whole space $W$.

Let $\Vec{v}\in V$. Then, by definition of vector space, there exists $-\Vec{v}$ such that $\Vec{v}+(-\Vec{v})=\Vec{0_V}$. Now, since $T$ is linear, $$T(\Vec{0_V})=T[\Vec{v}+(-\Vec{v})]=T(\Vec{v})+T(-\Vec{v})$$

zero vector, $\Vec{v}+\Vec{0_V}=\Vec{v}$. Now, since $T$ is linear, we have $$T(\Vec{v})=T(\Vec{v}+\Vec{0_V})=T(\Vec{v})+T(\Vec{0_V}).$$