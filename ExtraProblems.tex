\documentclass{exam}
\usepackage[all]{xy}
\usepackage{color,graphicx}
\usepackage{esvect}
\usepackage{amsfonts, amssymb,amsmath}
\usepackage{enumitem}

\makeatletter
\renewcommand*\env@matrix[1][*\c@MaxMatrixCols c]{%
  \hskip -\arraycolsep
  \let\@ifnextchar\new@ifnextchar
  \array{#1}}
\makeatother


\newcommand\Rn{$\mathbb{R}^n$}
\newcommand\Rm{$\mathbb{R}^m$}
\newcommand\R{$\mathbb{R}$}
\newcommand\bq{\begin{question}}
\newcommand\eq{\end{question}}
\newcommand\be{\begin{enumerate}}
\newcommand\ee{\end{enumerate}}
\renewcommand{\labelenumi}{\alph{enumi})}
\newcount\colveccount
\newcommand*\colvec[1]{
        \global\colveccount#1
        \begin{bmatrix}
        \colvecnext
}
\def\colvecnext#1{
        #1
        \global\advance\colveccount-1
        \ifnum\colveccount>0
                \\
                \expandafter\colvecnext
        \else
                \end{bmatrix}
        \fi
}

\begin{document}

\textbf{Extra Problems}: 
\begin{enumerate}
\item[A.] Give an example of an invertible linear transformation from $\mathbb{R}^3\to\mathbb{R}^3$.

Some examples include the identity transformation $$T_{Id}\left(\colvec{3}{x}{y}{z}\right)=\colvec{3}{x}{y}{z},$$ the projection which scales every vector by a factor of 2 $$T_2\left(\colvec{3}{x}{y}{z}\right)=\colvec{3}{2x}{2y}{2z},$$ and $$T_B\left(\colvec{3}{x}{y}{z}\right)=\colvec{3}{-7x-6y-12z}{5x+5y+z}{x+4z}.$$ % FCLA Archetype B

You should verify on your own that these are linear transformations and that they are invertible.

\item[B.] Give an example of a linear transformation from $\mathbb{R}^3\to\mathbb{R}^3$ that is not invertible.

Some examples include the zero transformation $$T_0\left(\colvec{3}{x}{y}{z}\right)=\colvec{3}{0}{0}{0},$$ the projection onto the $xy-$plane $$T_{xy}\left(\colvec{3}{x}{y}{z}\right)=\colvec{3}{x}{y}{0},$$ and $$T_A\left(\colvec{3}{x}{y}{z}\right)=\colvec{3}{x-y+2z}{2x+y+z}{x+y}.$$ % FCLA Archetype A

You should verify on your own that these are linear transformations and that they are not invertible. 

\item[C.] Suppose that $Q$ is an invertible $n\times n$ matrix and $D$ is a diagonal matrix. Describe the matrix $(QDQ^{-1})^k$.

Notice that $$(QDQ^{-1})^2=(QDQ^{-1})(QDQ^{-1})=QD(Q^{-1}Q)DQ^{-1}=QD(Id_n)DQ^{-1}=QDDQ^{-1}=QD^2Q^{-1},$$ which leads to the following:

\textbf{Conjecture:} For all $k\in\mathbb{N}$, $(QDQ^{-1})^k=QD^kQ^{-1}$.

\textit{Proof}: We proceed by induction. The base case is trivial since $(QDQ^{-1})^1=QDQ^{-1}=QD^1Q^{-1}$.

For the inductive hypothesis, let $k\in\mathbb{N}$ and assume that $(QDQ^{-1})^k=QD^kQ^{-1}$. We must show that $(QDQ^{-1})^{k+1}=QD^{k+1}Q^{-1}$. Now, using the inductive hypothesis,
\begin{align*}
(QDQ^{-1})^{k+1} & =(QDQ^{-1})^k(QDQ^{-1}) \\ 
& = (QD^kQ^{-1})(QDQ^{-1}) \\
& = QD^k(Q^{-1}Q)DQ^{-1} \\ 
& = QD^k(Id_n)DQ^{-1} \\ 
& = QD^kDQ^{-1} \\ 
& = QD^{k+1}Q^{-1} \\
\end{align*}
as desired. Therefore, by the Principle of Mathematical Induction, $(QDQ^{-1})^k=QD^kQ^{-1}$ for any natural number $k$.

\end{enumerate}\newline
\vspace{0.1in}
\newline



\end{document}