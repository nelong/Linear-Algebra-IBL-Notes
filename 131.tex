\documentclass{exam}
\usepackage[all]{xy}
\usepackage{color,graphicx}
\usepackage{esvect}
\usepackage{amsfonts, amssymb,amsmath}
\usepackage{enumitem}

\makeatletter
\renewcommand*\env@matrix[1][*\c@MaxMatrixCols c]{%
  \hskip -\arraycolsep
  \let\@ifnextchar\new@ifnextchar
  \array{#1}}
\makeatother

\newcommand\Rn{$\mathbb{R}^n$}
\newcommand\Rm{$\mathbb{R}^m$}
\newcommand\R{$\mathbb{R}$}
\newcommand\bq{\begin{question}}
\newcommand\eq{\end{question}}
\newcommand\be{\begin{enumerate}}
\newcommand\ee{\end{enumerate}}
\renewcommand{\labelenumi}{\alph{enumi})}
\newcommand*\colvec[1]{
        \global\colveccount#1
        \begin{bmatrix}
        \colvecnext
}
\def\colvecnext#1{
        #1
        \global\advance\colveccount-1
        \ifnum\colveccount>0
                \\
                \expandafter\colvecnext
        \else
                \end{bmatrix}
        \fi
}

\begin{document}


\textbf{Question 131}: Can you write $2+4t$ as a linear combination of $1+t$ and $-1+t$?\newline
\vspace{0.2in}
\newline

\textit{Proof}: We need to know whether the equation $$c_1(1+t)+c_2(-1+t)=2+4t$$ has a solution. Using the distributive property and collecting like terms gives the equation $$(c_1-c_2)+(c_1+c_2)t=2+4t.$$ Therefore, we must solve the system of equations
$$c_1-c_2=2$$
$$c_1+c_2=4$$
The augmented matrix corresponding to this system is \begin{bmatrix}[cc|c]
  1 & -1 & 2\\
  1 & 1 & 4
\end{bmatrix}
and its reduced row-echelon form is 
\begin{bmatrix}[cc|c]
  1 & 0 & 3\\
  0 & 1 & 1
\end{bmatrix}. Therefore, the system is consistent with the unique solution $c_1=3$ and $c_2=1$. We can verify that $$3(1+t)+1(-1+t)=2+4t,$$ and the answer is \textbf{YES}.

\end{document}